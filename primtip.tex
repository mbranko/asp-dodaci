\documentclass[11pt,a4paper]{article}

\usepackage{ifxetex}
\RequireXeTeX

\usepackage[a4paper,tmargin=1in,bmargin=1.3in,lmargin=1in,rmargin=1in]{geometry}

% bookmarks in PDF
\usepackage[colorlinks=true,linkcolor=blue,urlcolor=blue,bookmarksopen=true]{hyperref}
\usepackage[depth=3]{bookmark}
% open file with zoom to page width
\hypersetup{pdfstartview={FitH top}}

% math package
\usepackage{amsmath}

% customizing fonts
\usepackage{fontspec}
\usepackage{xcolor}
\usepackage{titlesec}
\setsansfont{Fira Sans}
\setmonofont{JetBrains Mono NL}[Scale=0.8]
\setmainfont{Dijakritika}
\titleformat*{\section}{\Large\bfseries\sffamily}
\titleformat*{\subsection}{\bfseries\sffamily}

% image handling
\usepackage{graphicx}
\usepackage{forest}
\usepackage{tikz}
\usetikzlibrary{arrows,petri,topaths,arrows.meta,automata,calc,shapes.multipart,chains,shapes.gates.logic.US,matrix}
\usepackage{tkz-berge}
\usepackage[position=top]{subfig}
\tikzstyle{branch}=[fill,shape=circle,minimum size=3pt,inner sep=0pt]

% syntax highligting
\usepackage{minted}

% custom headers and footers
\usepackage{fancyhdr}
\pagestyle{fancy}
\fancyhf{}
\chead{ASP primeri}
\cfoot{Strana \thepage}
\setlength{\headheight}{46pt}

\begin{document}

\section{Primitivni tipovi}

\subsection{Obrtanje redosleda bitova}

\textbf{Zadatak:} Napisati program koji uzima 64-bitni neoznačeni ceo broj i
vraća 64-bitni neoznačeni ceo broj koji se sastoji iz bita u obrnutom redosledu.
Na primer, ako je ulaz $(1110000000000001)_2$, izlaz bi trebalo da bude
$(1000000000000111)_2$.

\textbf{Savet:} Koristiti pomoćnu tabelu.

\textbf{Rešenje:} Ako je potrebno obaviti ovu operaciju samo jednom, postoji
jednostavno rešenje grubom silom: iteracija kroz 32 manje značajna bita i zamena
sa odgovarajućim bitom iz druge polovine ulaza.

Ako će se operacija izvršavati više puta, pažljivije ćemo analizirati strukturu
ulaza, imajući u vidu korišćenje keša. Neka se ulaz sastoji od četiri 16-bitna
cela broja $y_3$, $y_2$, $y_1$, $y_0$ gde $y_3$ sadrži najznačajnije bitove.
Najmanje značajnih 16 bita u obrnutom redosledu će doći iz $y_3$. Preciznije,
ovi bitovi će se pojaviti u obrnutom redosledu u odnosu na $y_3$. Na primer, ako
$y_3$ ima vrednost $(1110000000000001)_2$, onda će 16 najmanje značajnih bita u
konačnom rezultatu biti $(1000000000000111)_2$.

Slično računanju pariteta, vrlo brz način za obrtanje bitova u 16-bitnom celom
broju kada izvršavamo puno ovakvih operacija je izgradnja pomoćne tabele $A$ na
takav način da za svaki 16-bitni ceo broj $y$, $A[y]$ sadrži obrnute bite od
$y$. Obrtanje redosleda bitova u $x$ može se obaviti obrtanjem bita u $y_0$ za
najznačanije bite, zatim obrtanje redosleda u $y_1$, zatim u $y_2$, zatim u
$y_3$.

Ovakvo rešenje se može ilustrovati primerom obrtanja 8-bitnih celih brojeva uz
pomoćnu tabelu sa 2-bitnim ključevima. Tabela je $rev = ((00),(10),(01),(11))$.
Ako je ulaz $(10010011)_2$, rezultat je $rev(11),rev(00),rev(01),rev(10)$, tj.
$(11001001)_2$.

\begin{minted}[frame=lines]{python}
def reverse-bits(x):
    MASK_SIZE = 16
    BIT_MASK = 0xFFFF
    return (PRECOMPUTED_REVERSE[x & BIT_MIASK] << (3 * MASK_SIZE)
            | PRECOMPUTED_REVERSE[(x >> MASK-SIZE) & BIT_MASK] <<
            (2 * MASK-SIZE) |
            PRECOMPUTED_REVERSE [(x >> (2 * MASK_SIZE)) & BIT_MASK] << MASK_SIZE
            | PRECOMPUTED_REVERSE[(x >> (3 * MASK_SIZE)) & BIT_MASK])
\end{minted}

Vremenska složenost je $O(n/L)$, za $n$-bitne cele brojeve i $L$-bitne ključeve
za pomoćnu tabelu.

\subsection{Obrtanje redosleda cifara}

\textbf{Zadatak:} Napisati program koji uzima ceo broj i vraća ceo broj koji
sadrži dekadne cifre ulaznog broja u obrnutom redosledu. Na primer, obrtanjem 42
dobija se 24, a obrtanjem -314 dobija se -413.

\textbf{Savet:} Kako biste rešili problem ako bi ulaz bio prikazan kao string?

\textbf{Rešenje:} Pristup grubom silom bio bi konverzija ulaza u string, zatim
obrtanje znakova u stringu. Na primer, $(1100)_2$ je dekadni broj $12$,
a rezultat za $(1100)_2$ se može dobiti iteracijom kroz "12" u obrnutom
redosledu.

Detaljnija analiza pokazuje da se može izbeći formiranje stringa. Posmatrajmo
ulaz 1132. Prva cifra rezultata je 2, koja se može dobiti kao ostatak deljenja
ulaza sa 10. Preostale cifre se dobijaju od međurezultata 1132/10 = 113.
Uopšteno, neka je ulaz $k$. Ako je $k \geq 0$, onda $k \pmod{10}$ je
najznačajnija cifra rezultata a preostale cifre su rezultat obrtanja za $k/10$.
Nastavljajući sa primerom, iterativno ažuriramo međurezultat i ulaz kao 2 i 113,
zatim 23 i 11, zatim 231 i 1, zatim 2311.

U opštem slučaju za $k$, zapamtićemo znak, rešiti problem za $|k|$, i dodati
znak na kraju.

\begin{minted}[frame=lines]{python}
def reverse(x):
    result, x_remaining = 0, abs(x)
    while x_remaining:
        result = result * 10 + x_remaining % 10
        x_remaining //= 10
    return -result if x < 0 else result
\end{minted}

Vremenska složenost je $O(n)$, gde je $n$ broj dekadnih cifara u $k$.

\subsection{Presek pravougaonika}

Ovaj problem se odnosi na pravougaonike čije stranice su paralelne sa X i Y
osama. Slika \ref{fig:rectangles} ilustruje ove situacije.

\textbf{Zadatak:} Napisati program koji proverava da li dva pravougaonika imaju
neprazan presek. Ako imaju, vraća pravougaonik koji predstavlja njihov presek. 

\textbf{Savet:} Razmišljajte nezavisno o X i Y dimenzijama.

\textbf{Rešenje:} Kako zadatak to nije definisao, tretiraćemo stranicu kao deo
pravougaonika. To znači da, na primer, pravougaonici na slici
\ref{fig:rectangles} imaju presek.

\begin{figure}[hb]
\centering
\begin{tikzpicture}[scale=.75]
  \draw (0,0) rectangle (1,1) node[pos=.5] {A};
  \draw (1,1) rectangle (6,4) node[pos=.5] {B};
  \draw (4,0) rectangle (5,5) node[pos=.5] {C};
  \draw (7,0) rectangle (10,3.5) node[pos=.5] {D};
  \draw (10,0) rectangle (11,2) node[pos=.5] {E};
  \draw (9,3) rectangle (15,4) node[pos=.5] {F};
  \draw (13,3.1) rectangle (14,3.9) node[pos=.5] {G};
  \draw (14.5,2.5) rectangle (17,5) node[pos=.5] {H};
  \draw (17,2.5) rectangle (18,5) node[pos=.5] {I};
  \draw (17,1) rectangle (17.5,3) node[pos=.5] {J};
\end{tikzpicture}
\caption{Primeri XY-poravnatih pravougaonika}
\label{fig:rectangles}
\end{figure}

Ima puno načina kako se ovakvi pravougaonici mogu presecati. Na primer, mogu
imati delimično preklapanje ($D$ i $F$), jedan sadrži drugi ($F$ i $G$), mogu
deliti zajedničku stranicu ($D$ i $E$), mogu deliti zajedničko teme ($A$ i $B$),
mogu formirati krst ($B$ i $C$), ili oblik slova T ($F$ i $H$), itd. Mogućih
slučajeva ima zaista puno.

Bolji pristup je da se fokusira na uslove pod kojima je garantovano da se
pravougaonici ne seku. Na primer, pravougaonik sa temenom u levom donjem uglu
(1,2), širinom 3 i visinom 4 ne može se seći sa pravougaonikom čije teme u
donjem levom uglu je (5,3), širina 2 i visina 4, jer su vrednosti X koordinata
tačaka prvog pravougaonika u opsegu od 1 do 4 (uključivo) a X koordinate tačaka
drugog u opsegu od 5 do 7 (uključivo).

Slično tome, ako se Y koordinate tačaka prvog pravougaonika ne seku sa Y
koordinatama tačaka drugog, dva pravougaonika se ne seku.

Ekvivalentno tome, ako se opsezi X koordinata seku i opsezi Y koordinata seku,
sve tačke u tim presecima opsega pripadaju pravougaoniku koji predstavlja
neprazan presek.

\begin{minted}[frame=lines]{python}
Rectangle = collections.namedtuple('Rectangle', ('x', 'y', 'width', 'height'))

def intersect_rectangle(Rl, R2): 
    def is_intersect(R1, R2) :
        return (R1.x <= R2.x + R2.width and R1.x + R1.width >= R2.x
                and R1.y <= R2.y + R2.height and R1.y + R1.height >= R2.y)

    if not is_intersect(R1, R2):
        return Rectangle(0, 0, -1, -1) # No intersection.
    return Rectangle(
        max(R1.x, R2.x),
        max(R1.y, R2.y),
        min(R1.x + R1.width, R2.x + R2.width) - max(R1.x, R2.x), 
        min(R1.y + R1.height, R2.y + R2.height) - max(R1.y, R2.y))
\end{minted}

Vremenska složenost je $O(1)$, jer je broj operacija konstantan.

\textbf{Varijanta:} Ako su date četiri tačke u ravni, kako proveriti da li one
formiraju temena pravougaonika?

\textbf{Varijanta:} Kako proveriti da li se dva pravougaonika, čije stranice ne
moraju biti paralelne sa osama, seku? 

\end{document}