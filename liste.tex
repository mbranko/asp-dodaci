\documentclass[11pt,a4paper]{article}

\usepackage{ifxetex}
\RequireXeTeX

\usepackage[a4paper,tmargin=1in,bmargin=1.3in,lmargin=1in,rmargin=1in]{geometry}

% bookmarks in PDF
\usepackage[colorlinks=true,linkcolor=blue,urlcolor=blue,bookmarksopen=true]{hyperref}
\usepackage[depth=3]{bookmark}
% open file with zoom to page width
\hypersetup{pdfstartview={FitH top}}

% math package
\usepackage{amsmath}

% customizing fonts
\usepackage{fontspec}
\usepackage{xcolor}
\usepackage{titlesec}
\setsansfont{Fira Sans}
\setmonofont{JetBrains Mono NL}[Scale=0.8]
\setmainfont{Dijakritika}
\titleformat*{\section}{\Large\bfseries\sffamily}
\titleformat*{\subsection}{\bfseries\sffamily}

% image handling
\usepackage{graphicx}
\usepackage{forest}
\usepackage{tikz}
\usetikzlibrary{arrows,petri,topaths,arrows.meta,automata,calc,shapes.multipart,chains,shapes.gates.logic.US,matrix}
\usepackage{tkz-berge}
\usepackage[position=top]{subfig}
\tikzstyle{branch}=[fill,shape=circle,minimum size=3pt,inner sep=0pt]

% syntax highligting
\usepackage{minted}

% custom headers and footers
\usepackage{fancyhdr}
\pagestyle{fancy}
\fancyhf{}
\chead{ASP primeri}
\cfoot{Page \thepage}
\setlength{\headheight}{46pt}

\frenchspacing

\begin{document}

\section{Jednostruko spegnute liste}

Za sve zadatke u ovom odeljku, osim ako nije drugačije navedeno, svaki čvor ima
dva atributa -- polje sa podatkom i polje sa pokazivačem (adresom) na sledeći
čvor u listi. Polje za sledeći kod poslednjeg elementa ima vrednost
\texttt{None}. Čvor može biti predstavljen sledećom klasom:

\begin{minted}[frame=lines]{python}
class ListNode:
    def __init__(self, data=0, next_node=None)
        self.data = data 
        self.next = next_node  
\end{minted}  

Osnovne operacije nad listom -- pretraga, dodavanje i uklanjanje -- mogu se
implementirati na sledeći način.

\begin{minted}[frame=lines]{python}
def search_list(L, key): 
    while L and L.data != key:
        L = L.next
    # If key was not present in the List, L will have become null 
    return L


def insert_after(node, new_node):
    new_node.next = node.next
    node.next = new_node


# Delete the node past this one. Assume node is not a tail 
def delete_after(node):
    node.next = node.next.next
\end{minted}  

Ispod haube, Pajton tip \texttt{list} se tipično implementira kao niz sa
automatskom promenom veličine. Ovaj odeljak se bavi jednostruko i dvostruko
spregnutim listama koje nemaju standardni tip u Pajtonu. Definisaćemo sopstvene
tipove koje predstavljaju ovakve liste. Neke od operacije nad listama obuhvataju
vraćanje glave ili repa, dodavanje elementa na početak ili kraj liste, vraćanje
elementa koji se nalazi na početku ili kraju, uklanjanje sa početka ili kraja,
ili bilo kog elementa liste.

\subsection{Obrtanje podliste}

Ovaj problem se bavi obrtanjem podliste u okviru liste. Slike
\ref{fig:reversesublist1} i \ref{fig:reversesublist2} predstavljaju primer ove
operacije.

\begin{figure}[htb]
  \centering
  \begin{tikzpicture}[list/.style={
        rectangle split, 
        rectangle split parts=2,
        draw, 
        rectangle split horizontal,        
        prefix after command={\pgfextra{\tikzset{every label/.style={
          gray,
          below,
          yshift=-27,
          scale=0.6}}}}
      }, 
      >=stealth, 
      start chain, 
      pin edge = {
        Straight Barb-, 
        shorten <=1mm,semithick
      }
    ]
    \node[list,on chain,pin=180:L,label=0x2700] (A) {11};
    \node[list,on chain,label=0x2430] (B) {3};
    \node[list,on chain,label=0x1240] (C) {5};
    \node[list,on chain,label=0x1830] (D) {7};
    \node[list,on chain,label=0x1000] (E) {2};
    \draw[*->] let \p1 = (A.two), \p2 = (A.center) in (\x1,\y2) -- (B);
    \draw[*->] let \p1 = (B.two), \p2 = (B.center) in (\x1,\y2) -- (C);
    \draw[*->] let \p1 = (C.two), \p2 = (C.center) in (\x1,\y2) -- (D);
    \draw[*->] let \p1 = (D.two), \p2 = (D.center) in (\x1,\y2) -- (E);
  \end{tikzpicture}
  \caption{Početno stanje liste}
  \label{fig:reversesublist1}
\end{figure}

\begin{figure}[htb]
  \centering
  \begin{tikzpicture}[list/.style={
        rectangle split, 
        rectangle split parts=2,
        draw, 
        rectangle split horizontal,        
        prefix after command={\pgfextra{\tikzset{every label/.style={
          gray,
          below,
          yshift=-27,
          scale=0.6}}}}
      }, 
      >=stealth, 
      start chain, 
      pin edge = {
        Straight Barb-, 
        shorten <=1mm,semithick
      }
    ]
    \node[list,on chain,pin=180:L,label=0x2700] (A) {11};
    \node[list,on chain,label=0x1830] (D) {7};
    \node[list,on chain,label=0x1240] (C) {5};
    \node[list,on chain,label=0x2430] (B) {3};
    \node[list,on chain,label=0x1000] (E) {2};
    \draw[*->] let \p1 = (A.two), \p2 = (A.center) in (\x1,\y2) -- (D);
    \draw[*->] let \p1 = (D.two), \p2 = (B.center) in (\x1,\y2) -- (C);
    \draw[*->] let \p1 = (C.two), \p2 = (C.center) in (\x1,\y2) -- (B);
    \draw[*->] let \p1 = (B.two), \p2 = (D.center) in (\x1,\y2) -- (E);
  \end{tikzpicture}
  \caption{Lista sa obrnutom podlistom}
  \label{fig:reversesublist2}
\end{figure}

\textbf{Zadatak:} Napisati program koji dobija jednostruko spregnutu listu $L$ i
dva cela broja $s$ i $f$ kao argumente i obrće redosled čvorova u listi počevši
od $s$-tog čvora do $f$-tog čvora, uključivo. Čvorovi su numerisani počevši od
1, tj. glava liste ima redni broj 1. Nije dozvoljeno zauzimanje memorije za
dodatne čvorove.

\textbf{Savet:} Fokusirajte se na pokazivače na sledeći čvor koje treba
ažurirati.

\textbf{Rešenje:} Direktno rešenje je da se izdvoji podlista, obrne joj se
redosled, i vrati se nalaz u polaznu listu. Nedostatak ovog pristupa je da su
potrebna dva prolaska kroz podlistu.

Operacija se može izvesti u jednom prolazu kombinovanjem identifikacije podliste
sa njenim iteriranjem. Identifikovaćemo početak podliste iteriranjem kroz
čvorove do $s$-tog čvora. Kada dođemo do $s$-tog čvora, započinjemo obrtanje i
nastavljamo sa brojanjem. Kada dođemo do $f$-tog čvora, zaustavljamo obrtanje i
vezujemo obrnutu podlistu sa ostatkom početne liste.

\begin{minted}[frame=lines]{python}
def reverse_sublist(L, start, finish): 
    dummy_head = sublist_head = ListNode(0, L) 
    for _ in range(1, start):
        sublist_head = sublist_head.next

    # Reverses sublist
    sublist_iter = sublist_head.next 
    for _ in range(finish - start):
        temp = sublist_iter.next
        sublist_iter.next, temp.next, sublist_head.next = temp.next, sublist_head.next, temp
    return dummy_head.next  
\end{minted}  

Vremenska složenost zavisi od potrage za $f$-tim čvorom, tj. $O(f)$.

\textbf{Varijanta:} Napisati funkciju koja obrće jednostruko povezanu listu.
Operacija može da ima na raspolaganju samo konstantni dodatni memorijski prostor
($O(1)$).

\textbf{Varijanta:} Napisati program koji prima jednostruko spregnutu listu $L$
i nenegativni ceo broj $k$ i obrće listu po $k$ čvorova odjednom. Ako ukupan
broj čvorova u listi $n$ nije umnožak $k$, treba ostaviti poslednjih $n\mod k$
čvorova nepromenjenih. Ne treba menjati podatke koji se čuvaju u čvorovima.

\subsection{Test preklapanja listi}

Given two singly linked lists there may be list nodes that are common to both.
(This may not be a bug-it may be desirable from the perspective of reducing
memory footprint, as in the flyweight pattern, or maintaining a canonical
form.) For example, the lists in Figure \ref{fig:overlappinglists} overlap at 
Node D.

\begin{figure}[hb]
  \centering
  \begin{tikzpicture}[list/.style={
        rectangle split, 
        rectangle split parts=2,
        draw, 
        rectangle split horizontal,        
      }, 
      >=stealth, 
      start chain, 
      pin edge = {
        Straight Barb-, 
        shorten <=1mm,semithick
      },
      start chain=1,
      start chain=2
    ]
    \node[list,on chain=1,pin=180:L1] (A) {A};
    \node[list,on chain=1] (B) {B};
    \node[list,on chain=2,pin=180:L2] at (0,-1) (C) {C};
    \node[list,on chain=2] (D) {D};
    \node[list,on chain=2] (E) {E};
    \node[list,on chain=2] (F) {F};
    \draw[*->] let \p1 = (A.two), \p2 = (A.center) in (\x1,\y2) -- (B);
    \draw[*->] let \p1 = (C.two), \p2 = (C.center) in (\x1,\y2) -- (D);
    \draw[*->] let \p1 = (D.two), \p2 = (D.center) in (\x1,\y2) -- (E);
    \draw[*->] let \p1 = (E.two), \p2 = (E.center) in (\x1,\y2) -- (F);
    \draw let \p1 = (B.two), \p2 = (B.center) in (\x1,\y2) edge[*->,in=180,out=0,looseness=5] (D.north west);
  \end{tikzpicture}
  \caption{Primer preklapanja listi}
  \label{fig:overlappinglists}
\end{figure}

\textbf{Challenge:} Write a program that takes two cycle-free singly linked
lists, and determines if there exists a node that is common to both lists.

\textbf{Hint:} Solve the simple cases first.

\textbf{Solution:} We can avoid the extra space by using two nested loops, one
iterating through the first list, and the other to search the second for the
node being processed in the first list. However, the time complexity is
$O(n^2)$.

The lists overlap if and only if both have the same tail node: once the lists
converge at a node, they cannot diverge at a later node. Therefore, checking
for overlap amounts to finding the tail nodes for each list.

To find the first overlapping node, we first compute the length of each list.
The first overlapping node is determined by advancing through the longer list
by the difference in lengths, and then advancing through both lists in tandem,
stopping at the first common node. If we reach the end of a list without
finding a common node, the lists do not overlap.

\begin{minted}[frame=lines]{python}
def overlapping_no_cycle_lists(L1, L2):
    def length(L):
        length = 0 
        while L:
            length += 1
            L = L.next 
        return length

    L1_len, L2_len = length(L1), length(L2) 
    if L1_len > L2_len:
        L1, L2 = L2, L1  # L2 is the longer list
    # Advances the longer list to get equal length lists 
    for _ in range(abs(L1_len - L2_len)):
        L2 = L2.next

    while L1 and L2 and L1 is not L2: 
        L1, L2 = L1.next, L2.next

    return L1  # None implies there is no overlap between L1 and L2
\end{minted}

The time complexity is $O(n)$ and the space complexity is $O(1)$.

\subsection{Remove duplicates from a sorted list}

This problem is concerned with removing duplicates from a sorted list of
integers. See Figure \ref{fig:listduplicates} for an example of a list before
and after the removal of duplicates.

\begin{figure}[hb]
  \centering
  \begin{tikzpicture}[list/.style={
        rectangle split, 
        rectangle split parts=2,
        draw, 
        rectangle split horizontal,        
        prefix after command={\pgfextra{\tikzset{every label/.style={
          gray,
          below,
          yshift=-27,
          scale=0.6}}}}
      }, 
      >=stealth, 
      start chain, 
      pin edge = {
        Straight Barb-, 
        shorten <=1mm,semithick
      },
      start chain=1,
      start chain=2
    ]
    \node[list,on chain=1,pin=180:L,label=0x1000] (A) {2};
    \node[list,on chain=1,label=0x2110] (B) {2};
    \node[list,on chain=1,label=0x1830] (C) {3};
    \node[list,on chain=1,label=0x1240] (D) {5};
    \node[list,on chain=1,label=0x2200] (E) {7};
    \node[list,on chain=1,label=0x1200] (F) {11};
    \node[list,on chain=1,label=0x1354] (G) {11};
    \draw[*->] let \p1 = (A.two), \p2 = (A.center) in (\x1,\y2) -- (B);
    \draw[*->] let \p1 = (B.two), \p2 = (B.center) in (\x1,\y2) -- (C);
    \draw[*->] let \p1 = (C.two), \p2 = (C.center) in (\x1,\y2) -- (D);
    \draw[*->] let \p1 = (D.two), \p2 = (D.center) in (\x1,\y2) -- (E);
    \draw[*->] let \p1 = (E.two), \p2 = (E.center) in (\x1,\y2) -- (F);
    \draw[*->] let \p1 = (F.two), \p2 = (F.center) in (\x1,\y2) -- (G);

    \node[list,on chain=2,pin=180:L,label=0x1000] at (0,-2) (H) {2};
    \node[list,on chain=2,label=0x1830] (I) {3};
    \node[list,on chain=2,label=0x1240] (J) {5};
    \node[list,on chain=2,label=0x2200] (K) {7};
    \node[list,on chain=2,label=0x1200] (L) {11};
    \draw[*->] let \p1 = (H.two), \p2 = (H.center) in (\x1,\y2) -- (I);
    \draw[*->] let \p1 = (I.two), \p2 = (I.center) in (\x1,\y2) -- (J);
    \draw[*->] let \p1 = (J.two), \p2 = (J.center) in (\x1,\y2) -- (K);
    \draw[*->] let \p1 = (K.two), \p2 = (K.center) in (\x1,\y2) -- (L);
  \end{tikzpicture}
  \caption{An example of duplicate removal}
  \label{fig:listduplicates}
\end{figure}

\textbf{Challenge:} Write a program that takes as input a singly linked list
of integers in sorted order, and removes duplicates from it. The list should
be sorted.

\textbf{Hint:} Focus on the successor fields which have to be updated.

\textbf{Solution:} A brute-force algorithm is to create a new list, using a
hash table to test if a value has already been added to the new list.
Altematively, we could search in the new list itself to see if the candidate
value already is present. If the length of the list is $n$, the first approach
requires $O(n)$ additional space for the hash table, and the second requires
$O(n^2)$ time to perform the lookups. Both allocate $n$ nodes for the new
list.

A better approach is to exploit the sorted nature of the list. As we traverse
the list, we remove all successive nodes with the same value as the current
node.

\begin{minted}[frame=lines]{python}
def remove_duplicates(L):
    it = L
    while it:
        # Uses next_distinct to find the next distinct value 
        next_distinct = it.next
        while next_distinct and next_distinct.data == it.data:
            next_distinct = next_distinct.next 
        it.next = next_distinct
        it = next_distinct
    return L  
\end{minted}

Determining the time complexity requires a little amortized analysis. A single
node may take more than $O(1)$ time to process if there are many successive
nodes with the same value. A clearer justification for the time complexity is
that each link is traversed once, so the time complexity is $O(n)$. The space
complexity is $O(1)$.

\textbf{Variant:} Let $m$ be a positive integer and $L$ a sorted singly linked
list of integers. For each integer $k$, if $k$ appears more than $m$ times in
$L$, remove all nodes from $L$ containing $k$.

\end{document}